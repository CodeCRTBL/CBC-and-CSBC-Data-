Morgan fingerprints were used to relate molecular structure with bioactivity against HT-116. The extracted structural fragments using MFPs were either directly counted in CBC, or counter only after pre-clustering in CSBC. Clustering in CSBC allowed for the identification of common bits that either positively contribute, negatively contribute, or do not contribute to bioactivity against HT-116. Direct counting of molecular fragments (bits) in CBC necessarily viewed the MFPs as being linearly correlated to bioactivity with the presence or absence of bits assumed to predict bioactivity. This led to at an average accuracy of 70\% together with a specificity, F1, and AUC score of 72\%. This suggests that CBC can only moderately predict bioactivity against HT-116 and may not scale linearly with increasing number of unknowns. In contrast, CSBC appeared to have consolidated common features among the clusters representing molecules having high inhibition, low inhibition, and no-inhibition to HT-116. The results suggests that features related to position and structural environment get to be folded in to the classification from the clustering stage.

The 3 models XGB, RF, and SVM when used with CSBC showed an average accuracy of 94\% , recall, f1 and AUC scores of 98\%. This indicates that these models have the necessary features to afford excellent and stable classification of activity. The performance of the models may be attributed to the clustering of common features in CSBC and with the self-correcting characteristics of the mathematical models, which enables them to fine-tune on the common features grouped in CSBC generated cluster classes.

%The MFP of a molecule can used as a key indicator of its bioactivity. In this study, it is extracted from the molecules 2D-structure via CBC and CSBC methods. In CBC, it assumes that the MFP are linearly correlated with the bioactivity, thus, the absence and presence of it were hypothesized to predict the molecules bioactivity against HCT-116. Results revealed that, CBC-ML has an accuracy of 70\% , specificity and F1 and AUC score of 72\%. Although, it can be considered as a good model, it is expected that if the unseen data increases, likely these parameters will decrease. This results implicitly suggest that significant bits from CBC are not linearly correlated to \% Inhibition against HCT-116. On the other hand, CSBC accounts for the position and neighbor dependency of the bits. The top 3 models (XGB, RF and SVM) of CSBC-ML show an average accuracy of 94\% , recall, f1 and AUC scores of 98\%. This results suggest that these models are excellent and stable on classifying molecules activity. This performance of the model was attributed to the following: a) CSBC bits position and neighbor dependency; b) self-correcting features of the mathematical model, which enables them to perceive the bits dependency on position and neighbor.  

Overall, the study was able to show that activity and inactivity molecules against HCT-116 can be predicted based on MFP fragments, excellent models can be created solely based on rigorous selection of MFPs, and CSBC-ML models have a promising future in virtual drug screening. 

