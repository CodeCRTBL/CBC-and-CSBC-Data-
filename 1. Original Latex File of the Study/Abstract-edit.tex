In this work we present a method for virtual drug screening based on the use of Morgan Fingerprinting (MFP) and Machine Learning (ML). Two models were created to classify molecular activity against HCT-116. Either model relies on counting of binary-represented molecular fragments generated from MFP, referred to as bits. In Crude Bit Counting (CBC), bits are identified and counted directly based on their absence or presence, while a counterpart method, Cluster-Subtraction Bit Counting (CSBC) pre-classified molecules based on relative activity against HT-116 through clustering before counting bits among clusters. Results show that after coupling with Machine Learning methods (ML), CSBC outperforms CBC based on various parameters pertaining to classification. CSBC-ML has an average accuracy of 94\%, and 98\% for recall, f1 and Area-Under-the-Curve (AUC) score during its training and testing on classifying activity or inactivity against HCT-116. In addition, its ROC curves shows no over-fitting, suggesting robustness. Further inspection of structures classified by CSBC suggests that the clustering stage allows for the consolidation of bits that have similar position and structural environment. These results demonstrated that MFPs generated via CSBC are the best features to use in classifying molecular activity against HCT-116, and that CSBC-ML is a promising method for applications in quantitative structure activity relationship (QSAR).





