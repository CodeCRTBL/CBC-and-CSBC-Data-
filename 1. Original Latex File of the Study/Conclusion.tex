The MFP of a molecule can used as a key indicator of its bioactivity. In this study, it is extracted from the molecules 2D-structure via CBC and CSBC methods. In CBC, it assumes that the MFP are linearly correlated with the bioactivity, thus, the absence and presence of it were hypothesized to predict the molecules bioactivity against HCT-116. Results revealed that, CBC-ML has an accuracy of 70\% , specificity and F1 and AUC score of 72\%. Although, it can be considered as a good model, it is expected that if the unseen data increases, likely these parameters will decrease. This results implicitly suggest that significant bits from CBC are not linearly correlated to \% Inhibition against HCT-116. On the other hand, CSBC accounts for the position and neighbor dependency of the bits. The top 3 models (XGB, RF and SVM) of CSBC-ML show an average accuracy of 94\% , recall, f1 and AUC scores of 98\%. This results suggest that these models are excellent and stable on classifying molecules activity. This performance of the model was attributed to the following: a) CSBC bits position and neighbor dependency; b) self-correcting features of the mathematical model, which enables them to perceive the bits dependency on position and neighbor.  

Overall, the study were able to show that: a) activity and inactivity molecules bioactivity against HCT-116 can be predicted based on MFP's motifs; b) excellent models can be created solely based on rigorous selection of MFP; c) CSBC-ML models can go  further robust testing and be use in virtual drug screening. 

