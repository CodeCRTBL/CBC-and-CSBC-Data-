Combination of Quantitative Structure-Activity Relatioship (QSAR) and Machine Learning (ML) offers a promising approach on virtual drug screenings, however, building one with good predicting capacity still poses as a challenge. In this study, two ML models are created to classify molecules activity and inactivity against HCT-116 --- CBC-ML and CSBC-ML . This is made possible through groundwork creation of selecting significant molecular fingerprints (MFP) --- substructure of molecules that might be responsible for its activity, via Crude Bits Counting (CBC) and Cluster-Subtraction Bits Counting (CSBC). Both CBC and CSBC combines Morgan Fingerprinting (MFA) and bit counting, however, the latter uses clustering and structure subtraction. Results shows that CSBC-ML outperforms CBC-ML models based on confusion matrix parameters (accuracy, f1 score and recall), Area Under the Curve (AUC) and Receiver Operating Characteristic (ROC) curve. Results show that CSBC-ML has an average accuracy of 94\%, and 98\% for recall, f1 and AUC score during its training and testing on classifying compounds activity or inactivity against HCT-116. In addition, its ROC curves shows no over-fitting, suggesting robustness against unseen data. The excellent performance of the model were attributed to the selection of features in which its position and neighbors were inherently accounted, and the self-correcting features of the mathematical model used. These results suggest that MFPs generated via CSBC are the best features to use in classifying molecules activity against HCT-116, and CSBC-ML models can be used for virtual drug screening.    




